% --------------------------------------------------
% Fájl: 01_commands.tex
% --------------------------------------------------

% --------------------------------------------------
% Matematikai parancsok
% --------------------------------------------------

% Általános matematikai parancsok
\renewcommand{\d}[1]{\mathrm{d}#1} % differenciál jel
\renewcommand{\vec}[1]{\mathbf{#1}} % vektor jelölés
\newcommand{\norm}[1]{\left\lVert #1 \right\rVert} % norma
\newcommand{\abs}[1]{\left\lvert #1 \right\rvert} % abszolút érték

% Lineár algebrai műveletek
\DeclareMathOperator{\rank}{rank} % rang
\DeclareMathOperator{\tr}{tr} % nyom (trace)
\DeclareMathOperator{\diag}{diag} % diagonális mátrix

% Vektoranalízis műveletek
\DeclareMathOperator{\grad}{\mathbf{grad}} % grad
\DeclareMathOperator{\rot}{\mathbf{rot}} % rot
\DeclareMathOperator{\diverg}{\mathbf{div}} % div
\DeclareMathOperator{\laplace}{\Delta} % Laplace-operátor

% Egyéb műveletek
\newcommand{\eval}[2]{\left. #1 \right|_{#2}} % kifejezés kiértékelése

% --------------------------------------------------
% Egyéb parancsok
% --------------------------------------------------

\newcommand*\circled[1]{\begin{tikzpicture}[baseline=(C.base)] \node[draw,circle,inner sep=2pt](C) {#1}; \end{tikzpicture}} % karikázás