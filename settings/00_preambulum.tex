% --------------------------------------------------
% Fájl: 00_preambulum.tex
% --------------------------------------------------

% --------------------------------------------------
% Packagek
% --------------------------------------------------

% Dokumentum konfiguráció
\usepackage[hungarian]{babel} % dokumentum nyelvi beállítása
\usepackage[utf8]{inputenc} % bemeneti betűkódolás (ékezetes karakterek)
\usepackage[T1]{fontenc} % kimeneti betűkódolás (ékezetes karakterek)
\usepackage{lmodern} % bővített karakterkészlet az ékezetekhez
\usepackage{geometry} % dokumentum margók és oldalbeállítások
\usepackage{hyperref} % hivatkozások és linkek
\usepackage[backend=biber,style=numeric,sorting=none]{biblatex} % irodalomjegyzék

% Ábrák, grafikák
\usepackage{graphicx} % klépek beillesztése és méretezése
\usepackage{float} % ábrák és táblázatok elhelyezésének pontos szabályozása
\usepackage{caption} % képaláírások formázása
\usepackage{tikz} % vektorgrafikák, ábrák rajzolása LaTeX-ben
\usepackage{xcolor} % színek kezelése
\usepackage{pdfpages} % teljes PDF-oldalak beillesztése a dokumentumba

% Matematikai és fizikai írásmód
\usepackage{amsmath,amsthm,amsfonts,amssymb,amscd} % matematikai környezetek és szimbólumok
\usepackage{mathtools} % kiegészítő matematikai eszközök az amsmath-hoz
\usepackage{siunitx} % SI mértékegységek helyes jelölése

% Fejléc, lábléc
\usepackage{fancyhdr} % egyedi fejléc és lábléc készítése
\usepackage{lastpage} % az utolsó oldal sorszámának hivatkozása

% Dokumentumtagolás
\usepackage{titlesec} % címsorok formázásának testreszabása
\usepackage{csquotes} % idézetek helyes formázása (nyelvspecifikus)
\usepackage{booktabs} % szép, tipográfiailag helyes táblázatok készítése


% --------------------------------------------------
% Dokumentum beállítások
% --------------------------------------------------

% Dokumentum formátum
\geometry{a4paper,total={170mm,240mm},left=20mm,top=30mm} % oldalbeállítások: A4, margók méretezése
\setlength\parindent{0pt} % bekezdések behúzásának kikapcsolása

% Fejléc, lábléc beállítása
\pagestyle{fancy} % fancyhdr csomag stílus használata
\renewcommand{\headrulewidth}{0.4pt} % fejléc vonalvastagsága
\renewcommand{\footrulewidth}{0.4pt} % lábléc vonalvastagsága
\headheight 1.5em % fejléc magasságának beállítása
\rhead{\NEV{}, \NEPTUN{}} % jobb felső sarok
\chead{} % középső fejléc
\lhead{\TANTARGYKOD{}, \TANTARGY{}} % bal felső sarok
\lfoot{} % bal alsó sarok
\cfoot{} % középső lábléc
\rfoot{\small\thepage\ / \pageref{LastPage}} % jobb alsó sarok
\headsep 1.5em % fejléc és szöveg közötti távolság

% Hyperref beállítások - PDF megjelenítés
\hypersetup{
    colorlinks=true, % színes linkek
    linkcolor=blue, % belső hivatkozások színe
    urlcolor=cyan, % URL hivatkozások színe
    pdftitle={\CIM{}}, % PDF cím
    pdfauthor={\NEV{}}, % PDF szerző
    pdfsubject={\TANTARGY{} (\TANTARGYKOD{})}, % PDF tárgy
    bookmarksopen=true, % könyvjelzők alapból nyitva
    bookmarksnumbered=true, % könyvjelzők számozása
    pdfpagemode=UseOutlines % könyvjelzők megjelenítése PDF olvasóban
}

% Irodalomjegyék beállítása
\addbibresource{contents/literature.bib} % irodalomjegyzék forrásfájl

% Számozások beállítása fejezetenként
% \numberwithin{equation}{section} % egyenletek számozása fejezetenként
% \numberwithin{figure}{section} % ábrák számozása fejezetenként
% \numberwithin{table}{section} % táblázatok számozása fejezetenként

% Táblázat formátum beállítása - Függőlegesen középre igazított szöveg
\newcolumntype{L}[1]{>{\raggedright\let\newline\\\arraybackslash\hspace{0pt}}m{#1}} % balra zárt
\newcolumntype{C}[1]{>{\centering\let\newline\\\arraybackslash\hspace{0pt}}m{#1}} % középre zárt
\newcolumntype{R}[1]{>{\raggedleft\let\newline\\\arraybackslash\hspace{0pt}}m{#1}} % jobbra zárt
\renewcommand{\arraystretch}{1.25} % táblázatok sormagassága

% Mértékegységek formátuma
\sisetup{exponent-product = \cdot, per-mode=fraction} % SI jelölések

% Tikz
\usetikzlibrary{math} % matematikai számítások Tikz-ben
\usetikzlibrary{calc} % geometriai számítások (pl. koordináták)
\usetikzlibrary{angles} % szögjelölések rajzolása
\usetikzlibrary{quotes} % címkék és feliratok egyszerű kezelése
\usetikzlibrary{arrows.meta} % nyílstílusok



