\begin{tikzpicture}[
    % --- Saját stílusok definiálása ---
    vec/.style = {color=#1,very thick,-latex}, % vektor stílus, nyíllal
    link/.style = {color=#1, thick},           % vonal stílus
    point/.style = {circle,fill=black,inner sep = 1.5pt}, % csomópontok
    guide/.style = {color=#1,thin},            % segédvonal stílus
    coordSys/.style = {color=#1, thick,-latex} % koordinátarendszer nyilai
    ]

    % --- Paraméterek definiálása ---
    \def\r{2}        % sugár
    \def\phi{60}     % szög (fokban)

    % --- Koordináták megadása ---
    \coordinate (A) at (0,0);                            % A pont (origó)
    \coordinate (B) at (\r,\r);                          % B pont
    \coordinate (C) at (2*\r,0);                         % C pont
    \coordinate (O) at (\r,0);                           % O középpont
    \coordinate (P) at ({\r+\r*cos(\phi)},{\r*sin(\phi)}); % P pont (szög szerint)

    % --- Test (félkörív) kirajzolása ---
    \draw[link={black}] (A) arc (180:0:\r);

    % --- Erő vektora ---
    \draw[vec={green!50!black}] (C) -- node[midway, right] {$F$} +(0,-1);

    % --- Koordináta-rendszer kirajzolása ---
    \draw[coordSys={red}] (A) -- +(0,1) node[left] {$y$};  % y tengely
    \draw[coordSys={red}] (A) -- +(1,0) node[above] {$x$}; % x tengely

    % --- Támasz (háromszög és görgők) C pontnál ---
    \draw[link={black}, fill=white] (C) -- ($(C)+(0.35, 0.25)$) -- ($(C)+(0.35, -0.25)$) -- cycle;
    \foreach \i in {-0.15, 0, 0.15} {
        \draw[link={black}, fill=white] ($(C)+(0.425, \i)$) circle (0.075);
    }

    % --- Talaj ábrázolása A és C pont alatt ---
    \fill[gray!20!white] ($(A)+(-0.5,0)$) rectangle ($(A)+(0.5,-0.5)$);
    \draw[black,very thick] ($(A)+(-0.5,0)$) -- ($(A)+(0.5,0)$);
    \fill[gray!20!white] ($(C)+(0.5,0.5)$) rectangle ($(C)+(0.75,-0.5)$);
    \draw[black,very thick] ($(C)+(0.5,0.5)$) -- ($(C)+(0.5,-0.5)$);
    
    % --- Csomópontok megrajzolása ---
    \node[point] at (A) {};
    \node[point] at (B) {};
    \node[point] at (C) {};
    \node[above left] at (A) {$A$};
    \node[above] at (B) {$B$};
    \node[above left] at (C) {$C$};

    % --- Segédvonalak és sugár vektora ---
    \draw[guide={gray}, dashed] (A) -- (C);      % A és C közötti szaggatott vonal
    \draw[guide={gray}, dashed] (B) -- +(0,-\r); % B-ből függőleges segédvonal
    \draw[guide={blue}, -latex] (O) -- node[above left] {$R$} (P); % sugár vektor
\end{tikzpicture}
